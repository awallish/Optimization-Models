\documentclass{article}
\usepackage[utf8]{inputenc}

\title{A Dynamic Programming Algorithm for Optimization of Investment Project Allocation}
\author{Alex Wallish}
\date{May 2019}

\begin{document}

\maketitle

Abstract: 
Suppose one has $m$ dollars to invest among $n$ projects, and that investing $x$ in project $i$ yields an expected present value return of $f_{i}(x)$ and has volatility $v_{i}(x)$, for $ i = 1, ..., n. $ Given $f_{1}(x),...,f_{n}(x)$ and $v_{1}(x),...,v_{n}(x)$ and a volatility threshold, we present an algorithm for determining the vector $<x1,..,xi>$ which yields the maximum return while keeping the total volatility under the threshold.  



\pagebreak
\section{Introduction}
The problem of deciding what capital to allocate to what investment projects has many applications to both actuarial science and portfolio theory.  Often an investor or portfolio manager will have many possible investment vehicles available. An efficient algorithm for determining how much should be allocated into each investment project, so as to maximize return could prove to be very beneficial. In section 2 we will state a general problem, and give a DP algorithm for solving it.  In section 3, we will consider a more nuanced form of the problem and show how it could be beneficial.  In section 4 we will provide an algorithm for solving this problem.  And in section 5 we will make some observations about our algorithm. 

\section{General Problem}
Suppose that one has $m$ dollars to invest among $n$ projects and that investing $x$ in project $i$ yields a present value return of $f_{i}(x)$, $i = 1, ..., n$. The problem is to determine the integer amounts to invest in each project so as to maximize the sum of the returns.  That is, if we let $x_{i}$ denote the amount to be invested in project $i$, then our problem is to choose non-negative integers $x_{1}, ..., x_{n}$ such that $$\sum_{i=1}^{n}x_{i} = m$$ while maximizing $$\sum_{i=1}^{n} f_{i}(x_{i}).$$ Ross(source here) provides a general solution using dynamic programming:
\linebreak
Let $P_{j}(x)$ denote the maximal possible sum of returns when we have a total of $x$ to invest in projects $1, ..., j$. So, with this notation, $P_{n}(m)$ represents the maximal value of the problem posed. Determining $P_{n}(m)$, and the optimal investment amounts begins by finding the values of $P_{j}(x)$ for $x = 1, ..., m$, first for $j = 1$, then for $j=2$, and so on up to $j=n$.
\linebreak
Because the maximum return when $x$ must be invested in project 1 is $f_{1}(x)$, we have that $$P_{1}(x) = f_{1}(x).$$
Now, suppose that $x$ must be invested between projects 1 and 2. If we invest $y$ in project 2 then a total of $x-y$ is available to invest in project 1. Because the best return from having $x-y$ available to invest in project 1 is $P_{1}(x-y)$ it follows that the maximum sum of returns possible when the amount $y$ is invested in project 2 is $f_{2}(y) + P_{1}(x-y).$ As the maximum som of returns possible is obtained by maximizing the preceding over $y$, we see that 


\section{Stochastic Problem}

\section{Stochastic Solution}

\section{Observations}


\end{document}
